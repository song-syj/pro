\message{ !name(ps4hnd.tex)}%Warning: This was Latex'ed on a Mac in order to include the figures
% this next is to generate figures on the Mac
\def\picture #1 by #2 (#3){
  $${\vbox to #2{
    \hrule width #1 height 0pt depth 0pt
    \vfill
    \special{picture #3} % this is the low-level interface
    }}$$
  }

%\input  /zu/6001-devel/6001mac
\input 6001mac

\begin{document}

\message{ !name(ps4hnd.tex) !offset(-3) }


\psetheader{Fall Semester, 1993}{Problem Set 4}

\medskip

\begin{flushleft}
Issued:  Tuesday, September 28 \\
\smallskip
Tutorial preparation for: Week of October 4\\
\smallskip
Written solutions due: Friday, October 8 in Recitation \\
\smallskip
Reading: 
\begin{tightlist}
\item Course notes: finish section 2.2
\item code files {\tt hend.scm} and {\tt hutils.scm} (attached to this handout)
\end{tightlist}
\end{flushleft}

\begin{center}
{\bf A Graphics Design Language}
\end{center}

In this assignment, you will work with the Peter Henderson's
``square-limit'' graphics design language, which Hal described in
lecture on September 28.  Before beginning work on this programming
assignment, you should review the notes for that lecture.  The goal of
this problem set is to reinforce ideas about data abstraction and
higher-order procedures, and to emphasize the expressive power that
derives from appropriate primitives, means of combination, and means
of abstraction.
\message{ !name(ps4hnd.tex) !offset(611) }

\end{document}

